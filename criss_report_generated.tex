\documentclass[12pt,a4paper]{report}
\usepackage[margin=1in]{geometry}
\usepackage{graphicx}
\usepackage{hyperref}
\usepackage{fancyhdr}
\usepackage{array}
\usepackage{longtable}
\usepackage{tabularx}
\usepackage{lscape}
\usepackage[dvipsnames]{xcolor}
\usepackage{tcolorbox}
\usepackage{enumitem}
\usepackage{tikz}

% --- Watermark setup ---
\usepackage{transparent}
\usepackage{eso-pic}
\newcommand\BackgroundPic{
    \put(0,0){
    \parbox[b][\paperheight]{\paperwidth}{
      \vfill
      \centering
      \transparent{0.06}
      \resizebox{0.95\textwidth}{!}{
      \scalebox{2.1}{\Huge\sf CRISS ROBOTICS -- BITS Pilani}
      }
      \vfill
    }
  }
}

% --- Header & footer ---
\pagestyle{fancy}
\fancyhf{}
\fancyhead[L]{CRISS ROBOTICS REPORT}
\fancyhead[R]{ERC 2025}
\fancyfoot[C]{\thepage}

\begin{document}

\AddToShipoutPicture*{\BackgroundPic}

% -----------------------------------------------------------------
% COVER PAGE

\begin{titlepage}
    \centering
    {\Huge \bfseries ERC 2025 Rover Mission Report\par}
    \vspace{2cm}
    % \includegraphics[width=0.16\textwidth]{logo_placeholder}\\[1.5cm]
    \vspace{1.5cm}
    {\Large CRISS ROBOTICS\\BITS Pilani}\\[0.8cm]

    \begin{tabular}{rl}
            \textbf{Team Name:} & CRISS ROBOTICS, BITS Pilani \\
            \textbf{Competition:} & ERC 2025 Marsyard Challenge \\
            \textbf{Version:} & 1.0 (Auto-Generated) \\
            \textbf{Report Date:} & August 11, 2025 \\
            \textbf{Total Anomalies:} & 2 \\
    \end{tabular}

    \vfill
    \rule{0.8\textwidth}{0.8pt} \\
    \vspace{0.5cm}
    \itshape
    Mission exploration, anomaly detection, and scientific insight powered by CRISS ROBOTICS.
\end{titlepage}

\clearpage

% -----------------------------------------------------------------
% TABLE OF CONTENTS
{
    \hypersetup{linkcolor=black}
    \tableofcontents
    \thispagestyle{empty}
    \clearpage
}

% -----------------------------------------------------------------
% EXECUTIVE SUMMARY

\chapter{Executive Summary}
This report presents the automated analysis of 2 detected anomalies during the ERC 2025 Marsyard Challenge mission. The mission data has been processed and compiled to provide comprehensive documentation of all identified objects and their characteristics.

Key highlights:
\begin{itemize}
\item Total anomalies detected: 2
\item Automated classification and documentation completed
\item Detailed visual evidence and descriptions provided for each detection
\item Mission data successfully processed and analyzed
\end{itemize}

% -----------------------------------------------------------------
% MISSION OVERVIEW

\chapter{Mission Overview}
\section{Operational Context}
\vspace{0.5em}
\noindent \textbf{Competition Slot:} \underline{\hspace{4cm}}\\[0.5em]
\noindent \textbf{Marsyard/Field Description:} \\
\textcolor{gray}{Type your brief arena description here...} \\[0.8em]
\noindent \textbf{Mission Start Time:} \underline{\hspace{3cm}}\\[0.5em]
\noindent \textbf{Mission End Time:} \underline{\hspace{3cm}}\\[0.5em]
\noindent \textbf{Total Duration:} \underline{\hspace{2.5cm}} min

\section{Travel Time and Distance Stats}
\vspace{0.5em}
\noindent \textbf{Total Path Length:} \underline{\hspace{2.5cm}} meters\\[0.5em]
\noindent \textbf{Total Anomalies Detected:} 2\\[0.5em]
\noindent \textbf{Average Rover Speed:} \underline{\hspace{2.5cm}} m/s\\[0.5em]
\noindent \textbf{Number of Stops:} \underline{\hspace{2cm}}

% -----------------------------------------------------------------
% ROVER PATH SUMMARY

\chapter{Rover Path Summary}
\section{Path Visualization}
\vspace{0.7em}
Insert your MATLAB/Python path map or scan here.\\[1em]
\noindent
\begin{center}
\fbox{\parbox{0.82\textwidth}{
    \vspace{1.5cm}
    \centering
    \textit{Drag or insert your trajectory image or exported map here.}
    \vspace{1.5cm}
}}
\end{center}

\section{Waypoints Table}

\begin{center}
\begin{tabularx}{\textwidth}{|c|X|X|X|X|X|}
\hline
\textbf{\#} & \textbf{Description} & \textbf{Timestamp} & \textbf{X} & \textbf{Y} & \textbf{Z} \\
\hline
1 & Start Position & & & & \\
2 & Waypoint 1 & & & & \\
... &  &  &  &  &  \\
n & End Position & & & & \\
\hline
\end{tabularx}
\end{center}

% -----------------------------------------------------------------
% ANOMALY DETECTION & CLASSIFICATION

\chapter{Anomaly Detection \& Classification Summary}
\section{Master Anomaly Table}

\begin{center}
\begin{tabularx}{\textwidth}{|c|c|c|c|c|c|c|c|X|}
\hline
\textbf{\#} & \textbf{Anomaly ID} & \textbf{Type} & \textbf{Category} & \textbf{Subcategory} & \textbf{Coordinates (X,Y,Z)} & \textbf{Confidence} & \textbf{Duplication Status} & \textbf{Description/Notes} \\
\hline
1 & A001 & Helmet & USEFUL & general &   & 0.95 & None &   The helmet in the yellow box appears to be a pro... \\
\hline
2 & A002 & A & USEFUL & general &   & 0.95 & None &   rover, the object in the yellow box, is a vehicl... \\
\hline

\end{tabularx}
\end{center}

\section{Duplication Detection Table}
\begin{center}
\begin{tabularx}{0.8\textwidth}{|c|c|c|c|X|}
\hline
\textbf{Primary ID} & \textbf{Duplicate ID} & \textbf{Distance (m)} & \textbf{Resolution} & \textbf{Remarks} \\
\hline
 & & & & \\
 & & & & \\
\hline
\end{tabularx}
\end{center}

\section{Anomaly Category Distribution}
\vspace{0.5em}
Insert your pie/bar chart here for anomaly type and counts.\\
\begin{center}
\fbox{\parbox{0.7\textwidth}{\centering\vspace{1cm}\textit{Place chart or summary here.}\vspace{1cm}}}
\end{center}

% -----------------------------------------------------------------
% DETAILED ANOMALY REPORTS

\chapter{Detailed Anomaly Reports}


\vspace{2em}
\noindent
\textbf{Anomaly \#1 --- ID: A001 }

\begin{itemize}[leftmargin=1.7cm]
    \item \textbf{Type:} Helmet
    \item \textbf{Category:} USEFUL
    \item \textbf{Subcategory:} safety_equipment
    \item \textbf{Coordinates (X, Y, Z):}  
    \item \textbf{Detection Time:} \underline{\hspace*{3.5cm}}
    \item \textbf{Confidence:} \underline{\hspace*{2.3cm}}
    \item \textbf{Duplication Status:} \underline{\hspace*{3cm}}
\end{itemize}

\vspace{0.7em}
\noindent \textbf{Cropped Visual Evidence:}
\begin{center}
\fbox{
    \parbox{0.75\textwidth}{
        \centering
        \vspace{0.5cm}
        \textit{Image: Screenshot from 2025-06-17 20-11-03.png}
        \vspace{0.5cm}
        
        % Uncomment the line below and replace with actual image path if images are available
        % \includegraphics[width=0.7\textwidth]{/Users/todi/Documents/Code/Pythonprojects/CRISS/Probation/ProcessedImages/Screenshot from 2025-06-17 20-11-03.png}
        
        \vspace{1.5cm}
    }
}
\end{center}

\vspace{1em}
\noindent \textbf{Contextual Map/Location:}
\begin{center}
\fbox{
    \parbox{0.75\textwidth}{
        \centering
        \vspace{1.8cm}
        \textit{Insert anomaly map, cropped scan, or image.}
        \vspace{1.8cm}
    }
}
\end{center}

\vspace{1em}
\noindent \textbf{Detailed Description:} \\
  The helmet in the yellow box appears to be a protective gear item, likely used for exploration or survival in a harsh environment. Given its design and the barren, rocky landscape, it is indeed typical in a Mars environment, where such protective equipment would be essential for human survival.

\newpage

\vspace{2em}
\noindent
\textbf{Anomaly \#2 --- ID: A002 }

\begin{itemize}[leftmargin=1.7cm]
    \item \textbf{Type:} A
    \item \textbf{Category:} UNKNOWN
    \item \textbf{Subcategory:} general
    \item \textbf{Coordinates (X, Y, Z):}  
    \item \textbf{Detection Time:} \underline{\hspace*{3.5cm}}
    \item \textbf{Confidence:} \underline{\hspace*{2.3cm}}
    \item \textbf{Duplication Status:} \underline{\hspace*{3cm}}
\end{itemize}

\vspace{0.7em}
\noindent \textbf{Cropped Visual Evidence:}
\begin{center}
\fbox{
    \parbox{0.75\textwidth}{
        \centering
        \vspace{0.5cm}
        \textit{Image: Screenshot from 2025-06-17 20-11-10.png}
        \vspace{0.5cm}
        
        % Uncomment the line below and replace with actual image path if images are available
        % \includegraphics[width=0.7\textwidth]{/Users/todi/Documents/Code/Pythonprojects/CRISS/Probation/ProcessedImages/Screenshot from 2025-06-17 20-11-10.png}
        
        \vspace{1.5cm}
    }
}
\end{center}

\vspace{1em}
\noindent \textbf{Contextual Map/Location:}
\begin{center}
\fbox{
    \parbox{0.75\textwidth}{
        \centering
        \vspace{1.8cm}
        \textit{Insert anomaly map, cropped scan, or image.}
        \vspace{1.8cm}
    }
}
\end{center}

\vspace{1em}
\noindent \textbf{Detailed Description:} \\
  rover, the object in the yellow box, is a vehicle designed for exploring and traversing the surface of Mars. It is typical in a Mars environment, equipped with wheels, sensors, and instruments to navigate and collect data on the Martian terrain.

\newpage


% -----------------------------------------------------------------
% PATH AND TRAVEL ANALYSIS

\chapter{Path and Travel Analysis}
\section{Path Efficiency Charts}
\begin{center}
\fbox{
    \parbox{0.8\textwidth}{
        \centering
        \vspace{2cm}
        \textit{Insert your speed/time path charts here (MATLAB/Python).}
        \vspace{2cm}
    }
}
\end{center}

\section{Travel Summary Table}
\begin{center}
\begin{tabularx}{0.95\textwidth}{|l|X|}
\hline
\textbf{Metric} & \textbf{Value (to be typed)} \\
\hline
Total Distance Traveled         &  \\
Total Mission Duration (min)    &  \\
Average Speed (m/s)             &  \\
Number of Stops                 &  \\
Number of Anomalies Detected    & 2 \\
Detection Rate (per min)        &  \\
\hline
\end{tabularx}
\end{center}

% -----------------------------------------------------------------
% APPENDIX

\appendix

\chapter{Appendix: Full Coordinate Data}
\begin{landscape}
\begin{longtable}{|c|c|c|c|c|c|c|c|c|c|c|c|}
\hline
\textbf{Timestamp} & \textbf{Rover X} & \textbf{Rover Y} & \textbf{Rover Z} & \textbf{Anomaly ID} & \textbf{Anomaly X} & \textbf{Anomaly Y} & \textbf{Anomaly Z} & \textbf{Type} & \textbf{Category} & \textbf{Confidence} & \textbf{Notes} \\
\hline
% Add your full CSV data rows here if needed
\end{longtable}
\end{landscape}

\chapter{Appendix: Scripts Used}
\begin{tcolorbox}[colback=gray!10!white, colframe=gray!80!black, title=Python Processing Script]
\begin{verbatim}
import pandas as pd
df = pd.read_csv('data.csv')
anomalies = df[df['object_identified'].notna()]
# Automated processing and LaTeX generation completed
print(f"Total anomalies processed: {len(df)}")
\end{verbatim}
\end{tcolorbox}

\chapter{Authorship and Acknowledgements}
\noindent\textbf{Team:} CRISS Robotics, BITS Pilani\\
\noindent\textbf{Report Generated:} August 11, 2025\\
\noindent\textbf{Processing Method:} Automated CSV to LaTeX conversion\\

\vspace{0.8cm}
\noindent\textit{Brand watermark visible on every page.}

\end{document}